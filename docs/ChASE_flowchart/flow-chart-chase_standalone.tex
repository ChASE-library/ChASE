% Flowcharting techniques for easy maintenance
% Author: Brent Longborough
\documentclass[x11names]{article}
% \documentclass[preview,border=4mm,convert={density=300,size=1080x800,outext=.png}]{standalone}
\usepackage{tikz}
\usetikzlibrary{shapes,arrows,chains}
\usetikzlibrary{positioning,calc}
%%% <
% \usetikzlibrary{matrix}
% \pgfplotsset{compat=1.12}
% \usepgfplotslibrary{external} % creates a tight self-contained pdf figure for each tikzpicture
% \tikzexternalize % comment out to debug if latex errors: generates the external pdf
% \tikzset{external/force remake} % otherwise will use external pdf if it exists
% \tikzset{png export/.style={
% 	external/system call={
% 		pdflatex \tikzexternalcheckshellescape
% 			-halt-on-error -interaction=batchmode -jobname "\image" "\texsource";
% 		convert -units pixelsperinch -density 150 "\image.pdf" "\image.png";
% 		convert -units pixelsperinch -density 600 "\image.pdf" "\image-600.png";
% }}}
% \tikzset{png export}
% %\tikzsetexternalprefix{figures/} % output the pdf to an existing directory (needs to exist)
% \tikzsetnextfilename{myfigure}
\usepackage{xcolor}
\usepackage{verbatim}
\usepackage[active,tightpage]{preview}
\usepackage{amssymb}
% \usepackage[usenames,dvipsnames]{color}
\PreviewEnvironment{tikzpicture}
\setlength\PreviewBorder{5mm}%
%%%>
\begin{comment}
:Title: Easy-maintenance flowchart
:Tags: flowcharts
:Author: Brent Longborough
:Slug: flexible-flow-chart

  This TikZ example illustrates a number of techniques for making TikZ
  flowcharts easier to maintain:
    * Use of <on chain> and <on grid> to simplify positioning
    * Use of global <node distance> options to eliminate the need to 
      specify individual inter-node distances
    * Use of <join> to reduce the need for references to node names
    * Use of <join by> styles to tailor specific connectors
    * Use of <coordinate> nodes to provide consistent layout for
      parallel flow lines
    * A method for consistent annotation of decision box exits
    * A technique for marking coordinate nodes (for layout debugging)

    I encourage you to tinker at this file - add intermediate boxes,
    alter the global distance settings, and so on, to see how well (or
    ill!) it adapts.
\end{comment}



\begin{document}
% \begin{figure}[htbp]
% \centering
% =================================================
% Set up a few colours
% \colorlet{lcfree}{green}
% \colorlet{lcnorm}{blue}
% \colorlet{lccong}{red}
\definecolor{myorange}{HTML}{F1A340}
\definecolor{myblue}{HTML}{4183C4}
\definecolor{mylightblue}{HTML}{5999b5}
\definecolor{myviolet}{HTML}{998EC3}
\definecolor{mygray}{HTML}{666666}
\definecolor{mywhite}{HTML}{f1f1f1}
\colorlet{lcfree}{mygray}
\colorlet{lcnorm}{mygray}
\colorlet{lccong}{myorange}
\colorlet{lcorange}{myorange}
\colorlet{lclblue}{mylightblue}
\colorlet{lcviolet}{myviolet}
% -------------------------------------------------
% Set up a new layer for the debugging marks, and make sure it is on
% top
\pgfdeclarelayer{marx}
\pgfsetlayers{main,marx}
% A macro for marking coordinates (specific to the coordinate naming
% scheme used here). Swap the following 2 definitions to deactivate
% marks.
\providecommand{\cmark}[2][]{%
  \begin{pgfonlayer}{marx}
    \node [nmark] at (c#2#1) {#2};
  \end{pgfonlayer}{marx}
  } 
\providecommand{\cmark}[2][]{\relax} 
% -------------------------------------------------
% Start the picture
\tikzstyle{my above of} = [above=of #1.north]
\tikzstyle{my below of} = [below=of #1.south]
\begin{tikzpicture}[%
    >=triangle 60,              % Nice arrows; your taste may be different
    start chain=going below,    % General flow is top-to-bottom
%     node distance=6mm and 60mm, % Global setup of box spacing
    node distance=6mm and 40mm, % Global setup of box spacing
    every join/.style={norm},   % Default linetype for connecting boxes
    scale=0.9, every node/.style={scale=0.9}
    ]
% ------------------------------------------------- 
% A few box styles 
% <on chain> *and* <on grid> reduce the need for manual relative
% positioning of nodes
\tikzset{
  base/.style={draw, on chain, on grid, align=center, minimum height=4ex},
  proc/.style={base, rectangle, text width=16em},
  inout/.style={base, trapezium, align=left, text width=12em},
  test/.style={base, diamond, aspect=2, text width=7em},
  input/.style={inout, trapezium right angle=120, trapezium left angle=60},
  output/.style={inout, trapezium right angle=60, trapezium left angle=120},
  term/.style={base, rectangle, fill=mywhite, rounded corners, text width=10em},
  % coord node style is used for placing corners of connecting lines
  coord/.style={coordinate, on chain, on grid, node distance=6mm and 25mm},
  coord2/.style={coordinate},
  % nmark node style is used for coordinate debugging marks
  nmark/.style={draw, cyan, circle, font={\sffamily\bfseries}},
  % -------------------------------------------------
  % Connector line styles for different parts of the diagram
  norm/.style={->, draw, lcnorm},
  free/.style={->, draw, lcfree},
  cong/.style={->, draw, lccong},
  it/.style={font={\small\itshape}}
}
% -------------------------------------------------
% Start by placing the nodes
\node [term] (n1)     {Start};
\node [input, join, fill=myorange!80] (n2) {input: $N$, $A$
 \\ input:  \textsf{nev, nex, tol, deg}}; % \\ input: $\hat{V}$, \textsf{ritz} };
\node [proc, join, fill=myblue] (n3) {$\textsf{m}[\textsf{nev}]\leftarrow$ \textsf{deg} \\
  $\textrm{size}(\hat{X})\leftarrow 0$ \\ $(\tilde{\lambda}_1,
  \tilde{\lambda}_{\sf nev+nex}, \tilde{\lambda}_{N}, \hat{V})
  \leftarrow$ \textsc{lanczos}};
\node [proc, join, fill=lclblue, yshift=-1em] (n5)
{$\hat{V}\leftarrow$ \textsc{filter}($\hat{V}$,$\textsf{m}$)};
\node [proc, fill=lclblue, join=by lclblue, yshift=-1em] (n6)
{$\hat{Q}\leftarrow$ \textsc{orthonormalize}($\left[\hat{V}\
    \hat{X}\right]$)};
\node [proc, fill=lclblue, join=by lclblue, yshift=-1em] (n4)
{$(\hat{V},\tilde{\Lambda})\leftarrow$ \textsc{Rayleigh-Ritz}($A,\hat{Q}$)};
\node [proc, fill=lclblue, join=by lclblue, right=of n4, xshift=10em]
(n7) {$\textsf{res}\left[\ \right]\leftarrow$ \textsc{residuals}($\hat{V},\tilde{\Lambda}$)};
\node [proc, fill=lclblue, join=by lclblue, my above of=n7] (n8)
{($\hat{V},\Lambda, \hat{X})\leftarrow$
  \textsc{defl\&lock}($\hat{V},\tilde{\Lambda},\textsf{res}$) \\ $(\tilde{\lambda}_1,
  \tilde{\lambda}_{\sf nev+nex})\leftarrow $ (\textrm{min, max})$\left[\Lambda\ \tilde{\Lambda}\right]$};
\node [proc, fill=lclblue, join=by lclblue, my above of=n8] (n9)
{$\textsf{m}\leftarrow$ \textsc{degrees}(\textsf{tol},\textsf{res})  \\
  \textsc{sort}(\textsf{res},$\hat{V}, \tilde{\Lambda}; \textsf{m}$)};
\node [test, fill=mygray!70, join=by lclblue, my above of=n9] (n10) {$\textrm{size}(\hat{X})\geq\textsf{nev}$};
\node [output, my above of=n10, fill=myorange!80] (n11) {output: $(\hat{X},
  \Lambda)$, \textsf{res} \\ output: timers, decorators};
% \node [term,  my above of=n11, yshift=2em] (n11) {output $n$, $U$};
\node [term,  join, my above of=n11] (n12) {End};
% -------------------------------------------------
\node [coord, right=of n7] (c1)  {}; %\cmark{1}
\coordinate (M1) at ($(n4.south)!0.5!(n5.north)$);
% \node [coord] at (M1) (c2)  {}; \cmark{2}
\node [coord2] at (n7.north |- n5.east) (c2)  {}; %\cmark{2}
% \node [coord, my above of=n10] (c3)  {}; \cmark{3}
\node [coord, left=of n10] (c4)  {}; %\cmark{4}
% \node [coord, my below of=n6] (c5)  {}; \cmark{5}
\node [coord2] (c5) at (n7.north |- n6)  {}; %\cmark{5}
\coordinate (M2) at ($(n5)!0.5!(n10)$); %\node[draw,text width=4cm] at (M2) {M2};
\coordinate (M3) at ($(n2.south)!0.5!(n4.north)$); %\node[draw,text width=4cm] at (M3) {M3};


% -------------------------------------------------
% \draw [->,lcfree] (n6.east) -- (c5) -- (n7.south);

% \path (n7.east) to node [near start, yshift=1em] {$y$} (n8);
%   \draw [*->,lccong] (n7.east) -- (n8);
  
\path (n10.north) to node [near start, xshift=1em] {$yes$} (n11);
  \draw [*->,lcnorm] (n10.north) -- (n11);
  
% \path (n7.north) to node [near start, yshift=1em] {$n$} (n5); 
%   \draw [o->,lcfree] (n7.north) |- (c2) -- (n5.east);
  
\path (n10.west) to node [near start, yshift=1em] {$no$} (n5); 
  \draw [o->,lclblue] (n10.west) -- (c4) -| (M2 |- n10) -| (M2 |- n5.east)  -- (n5.east);

% -------------------------------------------------
\end{tikzpicture}
% \caption{Flowchart for the NEGF simulation. The inner self-consistency loop, visualized in violet, connects the Green's functions and the self energies, while the outer loop, visualized in orange, provides the update of the potential from the solution of the Possion equation.}
% \label{fig:flowchart}
% \end{figure}

\end{document}
